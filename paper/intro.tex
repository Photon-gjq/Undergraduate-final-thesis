\chapter{引言}

量子场论作为狭义相对论与量子力学的自然的结合,是目前最成功的描述微观世界的低能有效理论\parencite{weinberg1995quantum}。作为量子场论的时空对称群,洛伦兹群的不等价不可约$( A,B)$表示给出了场论中不同自旋的基本粒子,从而我们能够给出狭义相对论框架下的基本粒子的散射性质。同时,广义相对论是描述大尺度现象最成功的理论,因此考虑弯曲时空中,不同自旋场的传播是一个自然的问题。目前,自旋为$0,1/2,1$和$2$的场在弯曲时空中的动力学已经被很好地描述了,可见\parencite{chandrasekhar1998mathematical}等,而自旋为$3/2$的场,又称为Rarita-Schwinger场\parencite{rarita1941theory}(后称RS场)在弯曲时空中的行为却鲜有人知,而无质量的RS场更伴随着许多有意思的现象——如果把无质量的RS场作为物质场,那么它只能在里奇平坦的时空中存在\parencite{frauendiener_spin32_1995};而如果把无质量的RS场看作与引力相似的几何量,超引力的概念就自然而然浮现了\parencite{deser_consistent_1976,freedman_progress_1976}。然而,由于各个文献使用的符号体系大相径庭,研究方法也不尽相同。因此,本文旨在用较为现代的二分量旋量形式阐明这些问题,并给出一些必要的物理图像。

在使用二分量旋量形式重述广义相对论时,我们遵循Penrose,1984\parencite{penrose_spinors_1984}的思路,从闵氏时空的几何入手,考虑其光锥结构,从几何角度自然引出旋量的概念,随后用映射的方式严格定义旋量代数,并指出抽象指标这一符号系统的重要性,之后给出旋量代数与普通的矢量代数之间的关系,并给出通常矢量框架下的广义相对论中的结论。最后引入一般弯曲时空中的场,给出任意自旋的弯曲时空场方程。


\section{二分量旋量形式}

在爱因斯坦引力的框架下,时空是一个光滑的,局部平坦的四维流形\parencite{penrose_spinors_1984}。虽然没有全局的洛伦兹对称性,但局域的对称群仍然是洛伦兹群。时空流形的局部可以视为闵氏时空,而这个向量空间是洛伦兹群的$( 1/2,1/2)$表示的表示空间。实际上,如果从洛伦兹群的$( 1/2,0)$表示和$( 0,1/2)$表示以及它们对应的表示空间,即旋量空间出发,我们也可以重新构建作为$( 1/2,1/2)$表示空间的矢量空间。以此,我们可以用整套二分量旋量的语言重述广义相对论,同时这套方法能够带给我们以张量为基础的理论中难以得到的结论和图景,例如时空的因果结构\parencite{geroch1968spinor},扭量\parencite{penrose_spinors_1986,zichichi_twistors_1991}等。除此之外,以此建立的理论对于不同自旋的场来说形式上更为统一,因此,我们认为“旋量应当拥有比矢量更加基础的地位”。中文世界缺少对二分量旋量形式叙述的广义相对论的资料,因此,本文前半部分旨在将这些内容用中文重述,作为对于初学者有益的材料。主要框架取自Penrose,1984\parencite{penrose_spinors_1984}。虽然本文主体与\parencite{penrose_spinors_1984}重合,但并不是在翻译原书,其中掺杂了不少个人见解以及诸多额外细节,同时也省去了许多与本文目的无关的内容,例如弯曲时空中的杨-米尔斯场,洛伦兹变换的旋量描述等。除此之外,其他关于二分量旋量形式的介绍或综述可见:\parencite{o2003introduction,alessio2018asymptotic}。

由于本文是初学者友好的,因此对本领域熟悉的读者可以跳过许多入门简介。本文中与主线关系不大的章节将加星号*表示,意为不是必读内容,但这些内容对理解二分量旋量形式也是必要的。


\section{彭罗斯-纽曼形式和微扰论}

本文的另一目的是探讨弯曲时空中的高自旋场的行为,即解弯曲时空中的高度非线性的场方程。然而而在较低的能量尺度下,物质场对背景引力场来说是小量,因此我们可以用微扰论来处理这个问题。使用旋量形式重述场方程后,可以将场方程用彭罗斯-纽曼(NP)形式表达,其好处是,原先非线性的二阶偏微分方程变为了一个对于微扰的一阶齐次线性微分方程组\parencite{chandrasekhar1998mathematical},在具体计算上更容易处理。对于自旋$1/2,1,2$的情况,钱德拉塞卡\parencite{chandrasekhar1998mathematical}已经讨论详尽,而自旋$3/2$的情况却鲜有文献涉及,这与自旋$3/2$在这种背景引力场下需要满足的自洽条件息息相关。本文会证明如果要求规范不变性且不考虑对背景引力场的反作用,那么仅在里奇平坦时空中,作为物质场的自旋$3/2$场的场方程才是一个良定义的柯西问题\parencite{frauendiener_spin32_1995}。在此之后会给出一些用彭罗斯-纽曼形式处理里奇平坦时空中的自旋$3/2$场微扰的例子。

需要再度声明的是,本文的前半部分内容虽然大量借鉴了Penrose,1984\parencite{penrose_spinors_1984},但并不是在单纯抄写翻译。同时,相似内容笔者也不会再度引用,大部分无引用内容在\parencite{penrose_spinors_1984}中都可以找到。第七章则是作者个人的工作,因此第七章中的内容都会标注来源。